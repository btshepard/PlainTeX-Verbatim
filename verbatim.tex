% \begindisplay ... \enddisplay for displaying text
\def\begindisplay{\begingroup\par\ \par\hfil}
\def\enddisplay{\hfil\par\ \par\endgroup}

% \def\begindisplay{\par\ \par\begingroup\leftskip=0pt plus 1fil\rightskip=\leftskip\parindent=0pt\parfillskip=0pt}
% \def\enddisplay{\par\ \par\endgroup}

% |...| and ||...|| for inline/display verbatim
% \def\showspaces{\catcode`\ =12\relax}
\newif\ifblankline
\def\obeyblanklines{\begingroup\lccode`~=`\^^M\relax\lowercase{\endgroup\def~}{\endgraf\ifblankline\vskip\baselineskip\fi\blanklinetrue}\catcode`\^^M=\active\everypar{\blanklinefalse}}
\def\disobeyblanklines{\catcode`\^^M=5\relax}
\catcode`|=13
\newif\ifvsp\vsptrue
\def\showspaces{\begingroup\lccode`~=` \lowercase{\endgroup\obeyspaces\let~=\ \relax}}
\def\makeinnocent#1{\catcode`#1=12\relax}
\def|{\begingroup\let\do\makeinnocent\dospecials\ifvsp\obeyspaces\fi\obeylines\startverb}
\def\startverb#1{\ifx|#1\relax\catcode`|=12\relax\expandafter\displayverb\else\inlineverb#1\fi}
\long\def\inlineverb#1|{\tt#1\endgroup\vsptrue}
\edef\tmp{\long\def\noexpand\displayverb##1\string|\string|}
\tmp{\begindisplay\showspaces\obeyblanklines\par\vskip-\baselineskip\tt#1\enddisplay\endgroup\vsptrue}

% @|...| and @||...|| for inline/display verbatim with visible spaces
% \def\atchar{\char`\@\relax}
% \catcode`\@=13
% \def @#1{\ifx|#1\expandafter\exclm\else\atchar#1\fi}
% \def\exclm{\vspfalse|}

\def\gobble#1{}
\def\atchar{\char`\@\relax}
\catcode`\@=13
\begingroup
\gdef @{\futurelet\nextchar\doat}
\gdef\doat{\ifx\^^M\nextchar\expandafter\gobble\else\ifx|\nextchar\expandafter\vspfalse\else\atchar\fi\fi}
\endgroup

\def\makepipeother{\catcode`|=12\relax}
\def\makepipeactive{\catcode`|=13\relax}

\def\pipe{\bgroup\string|\egroup}

\centerline{\font\large=cmr10 at 18pt \large Plain\TeX\ Verbatim}
\vskip1.5\baselineskip{}
\centerline{Benjamin T. Shepard}
\vskip1ex{}
\centerline{June 29, 2022}
\vskip1.5\baselineskip{}
\openup0.4ex

An inline verbatim such as\ |this is \inline| can be made with the code {\tt\pipe...\pipe}. If visible spaces are desired, use 
{\tt @\string|...\string|} instead, which will produce @|this is \inline|. A display verbatim such as
||this is \display||
can be made with {\tt\pipe\pipe...\pipe\pipe}. If visible spaces are desired, use {\tt @\pipe\pipe...\pipe\pipe} instead, which will produce
@||this is \display.||
Every normal \TeX\ character (except {\tt\string|}) as well as all of the special characters can be used inside this environment, as shown in the following example:
||!$@$#%^&*()-+=\ /.,;'`~th@<is$ ]\bye *is{ #a '`1\rho > *\tes<t* \undefined .??[" ||
Normally, the pipe character {\tt\string|} can only be used inside the display verbatim {\tt\string|\string|...\string|\string|}. However, the macros |\makepipeother| and |\makepipeactive| turn the character {\tt\string|} into a normal character and back into an active character, respectively. This can be used as follows:\vskip-\baselineskip
\begindisplay
\tt
\makepipeother
\obeyblanklines
1 \string\makepipeother
2 || this is a | pipe ||
3 \string\makepipeactive
\disobeyblanklines
\makepipeactive
\enddisplay
\vskip-\baselineskip
There is also the macro |\pipe| which expands to
|| macro: -> \bgroup\string|\egroup ||
displaying the literal {\tt\string|}. It can be used to, for example, execute
|| {\tt\pipe\pipe...\pipe\pipe} ||
versus
\begindisplay
\tt
\makepipeother
\string\makepipeother\string{\string\tt||...||\string}\string\makepipeactive
\makepipeactive
\enddisplay
which will each display {\tt\pipe\pipe...\pipe\pipe}. Of course, this macro cannot be used inside the {\tt\pipe...\pipe} or {\tt\pipe\pipe...\pipe\pipe} environments.

The character {\tt @} is able to be used regularly, and can also be used at the end of lines. Code such as
||1 this is a t@st
2 @
3
4 this is a test@
5
6 \bye||
will compile as expected. These verbatim environments also obey spaces, lines, and blank lines.

\bye