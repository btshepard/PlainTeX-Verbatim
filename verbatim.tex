%%%% ============= TEX WIZARDRY ============ %%%%

% boilerplate
\long\def\gobble#1{}
\long\def\firstofone#1{#1}
\def\makeletter#1{\catcode`#1=11\relax}
\def\makeother#1{\catcode`#1=12\relax}
\def\makeactive#1{\catcode`#1=13\relax}
\def\ifnot#1{#1\else\expandafter\expandafter\fi\iffalse\iftrue\fi}

% display-verbatim boxes and lengths
\makeletter{@}
\newbox\tb@x
\def\settowidth#1#2{\setbox\tb@x\hbox{{#2}}#1\wd\tb@x\setbox\tb@x\box\voidb@x}
\makeother{@}
\newdimen\verblen
\newdimen\verbtw
\newdimen\mynegw\settowidth\mynegw{ }
\def\negspace{\hskip-\mynegw\relax}

% display environment
\def\begindisplay{\begingroup\vskip\baselineskip}
\def\enddisplay{\vskip\baselineskip\endgroup}

% safety macros
\def\makepipeother{\makeother{|}}
\def\makepipeactive{\makeactive{|}}
\def\pipe{\bgroup\string|\egroup}

% obey spaces/lines/blank lines
\def\showspaces{%
    \begingroup%
    \lccode`~=` \relax%
    \lowercase{\endgroup\obeyspaces\let~=\ \relax}%
}
\newif\ifblankline
\def\obeyblanklines{%
    \begingroup%
    \lccode`~=`\^^M\relax%
    \lowercase{\endgroup\def~}%
    {\endgraf\ifblankline\vskip\baselineskip\fi\blanklinetrue}%
    \makeactive{\^^M}%
    \everypar{\blanklinefalse}%
}

% define |...| and ||...|| verbatim environments
\newif\ifvisiblespaces
\makepipeactive
\def|{%
    \ifmmode%
        \char`|\makepipeother%
    \else%
        \begingroup%
        \let\do\makeother\dospecials%
        \ifnot\ifvisiblespaces\obeyspaces\fi%
        \obeylines%
        \expandafter\dopipe%
    \fi%
}
\def\dopipe#1{%
    \ifx|#1%
        \makepipeother\expandafter\displaypipe%
    \else%
        \inlinepipe#1\relax%
    \fi%
}

% |...| environment
\long\def\inlinepipe#1|{\tt#1\endgroup\visiblespacesfalse}

% ||...|| environment
\makepipeother
\long\def\displaypipe#1||{%
    \settowidth\verblen{#1}
    \verbtw=\hsize
    \advance\verbtw by -\verblen
    \divide\verbtw by 2
    \parindent=\verbtw
    \vskip\baselineskip%
    \showspaces%
    \obeyblanklines%
    \tt#1%
    \vskip\baselineskip%
    \endgroup%
    \visiblespacesfalse%
    \noindent\firstofone%
}
\makepipeactive

% @|...| and @||...|| for visible spaces
\def\atchar{\char`\@\relax}
\makeactive{@}
\begingroup
\gdef @{\futurelet\next\doat}
\gdef\doat{%
    \ifx\^^M\next%
        \expandafter\gobble%
    \else%
        \ifx|\next%
            \expandafter\visiblespacestrue%
        \else%
            \atchar%
        \fi%
    \fi%
}
\endgroup

% \bverb...\everb and \bverbatim...\everbatim
% as substitutes for |...| and ||...||
\def\bverb{%
    \begingroup%
    \makepipeother%
    \let\do\makeother\dospecials%
    \ifnot\ifvisiblespaces\obeyspaces\fi%
    \tt%
    \negspace\negthinspace\dobverb%
}
\def\bverbs{
    \visiblespacestrue%
    \begingroup%
    \makepipeother%
    \let\do\makeother\dospecials%
    \ifnot\ifvisiblespaces\obeyspaces\fi%
    \tt%
    \negspace\dobverbs%
}
\def\bverbatim{%
    \begingroup
    \makepipeother
    \let\do\makeother\dospecials%
    \ifnot\ifvisiblespaces\obeyspaces\fi%
    \showspaces%
    \obeyblanklines%
    \par\vskip-\baselineskip\relax%
    \begindisplay%
    \negspace\negthinspace\tt%
    \dobverbatim%
}
\begingroup
\catcode`\!=0
\makeother{\\}
!makeother{ }
!gdef!dobverbs #1\everbs{#1!everb}
!obeyspaces
!gdef!dobverb#1\everb{#1!everb}
!gdef!dobverbatim#1\everbatim{#1!everbatim}
!endgroup
\def\everb{\endgroup\visiblespacesfalse}
\def\everbatim{\enddisplay\endgroup\visiblespacesfalse}

%%%% ========= COMMENTS AND USAGE ========== %%%%

\centerline{\font\large=cmr10 at 18pt \large Plain\TeX\ Verbatim}
\vskip2\baselineskip{}
\centerline{Benjamin T. Shepard}
\vskip1ex{}
\centerline{July 9, 2022}
\vskip2\baselineskip{}
\openup0.4ex{}

An inline verbatim such as |this is \inline| can be made with the delimiters {\tt\pipe...\pipe}.
If visible spaces are desired, use 
\bverb @|...|\everb instead, which will produce @|this is \inline|.
A display verbatim such as
||this is \display||
can be made with {\tt\pipe\pipe...\pipe\pipe}.
If visible spaces are desired, use {\tt @\pipe\pipe...\pipe\pipe} instead, which will produce
@||this is \display.||
Every normal \TeX\ character (except {\tt\pipe}) as well as all of the special characters can be used inside this environment, as shown in the following example:
||!$@$#%^&*()-+=\ /.,;'`~th@<is$ ]\bye *{ #a '`1\rho > *\tes<t* \undefined .??["||
Normally, the pipe character {\tt\pipe} can only be used inside the display verbatim {\tt\pipe\pipe...\pipe\pipe}.
However, the macros |\makepipeother| and |\makepipeactive| turn the character {\tt\pipe} into an innocent character and into an active character, respectively.
There is also the macro |\pipe| which expands to {\tt\string\bgroup\string\string\pipe\string\egroup}.
These commands allow for use of the pipe literal {\tt\pipe} in horizontal mode. For example, both
||{\tt th\pipe{}s is a pipe}||
and
\begindisplay
\tt
\string\makepipeother\string{\string\tt\ th\string|s is a pipe\string}\string\makepipeactive
\enddisplay
will display {\tt th\pipe{}s is a pipe}.
Of course, these macros cannot be used inside the {\tt\pipe...\pipe} or {\tt\pipe\pipe...\pipe\pipe} environments.
In contrast, the character {\tt @} is able to be used normally, provided that it is not proceeded by the pipe character; \TeX\ will recognize the sequence {\tt @\pipe} as the beginning of a verbatim environment. 
These verbatim environments obey spaces, lines, and blank lines; this allows for code insertion such as
||void dfs(int p){
    if(o[p]) return;
    o[p]=1;
    
    c[p]=t++;
    for(int i=0; i<s[p].size(); ++i){
        dfs(s[p][i]);
    }
}||
which will compile as expected.

As a backup (in case {\tt\pipe} is needed), the four commands |\bverb...\everb| and |\bverbatim...\everbatim| are provided as substitutes for {\tt\pipe...\pipe} and {\tt\pipe\pipe...\pipe\pipe}. Furthermore, the commands |\bverb| and |\bverbs| gobble the first space of their argument. However, these environments cannot be used with |@|; instead, use |\bverbs...\everbs| and |\bverbatims...\everbatims|. Use this command in front of either environment for visible spaces. For example, the code
||\bverbs this is a \test\everbs||
will display \bverbs this is a \test\everbs.
These commands can also be used to display the {\tt\pipe} character: the code
||\bverb th|s is a pipe\everb||
will display \bverb th|s is a pipe\everb, as expected. When the |\bverb| macro is executed, \TeX\ turns every available character innocent until it scans the exact sequence |\everb|.

\vskip3ex{}
To fix: long lines of code do not wrap correctly.
||Oops, the very long line of code is here, it does not fit into the line width unfortunately.||

\bye